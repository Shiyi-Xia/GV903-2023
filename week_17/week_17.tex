\documentclass{beamer}\usepackage[]{graphicx}\usepackage[]{xcolor}
% maxwidth is the original width if it is less than linewidth
% otherwise use linewidth (to make sure the graphics do not exceed the margin)
\makeatletter
\def\maxwidth{ %
  \ifdim\Gin@nat@width>\linewidth
    \linewidth
  \else
    \Gin@nat@width
  \fi
}
\makeatother

\definecolor{fgcolor}{rgb}{0.345, 0.345, 0.345}
\newcommand{\hlnum}[1]{\textcolor[rgb]{0.686,0.059,0.569}{#1}}%
\newcommand{\hlstr}[1]{\textcolor[rgb]{0.192,0.494,0.8}{#1}}%
\newcommand{\hlcom}[1]{\textcolor[rgb]{0.678,0.584,0.686}{\textit{#1}}}%
\newcommand{\hlopt}[1]{\textcolor[rgb]{0,0,0}{#1}}%
\newcommand{\hlstd}[1]{\textcolor[rgb]{0.345,0.345,0.345}{#1}}%
\newcommand{\hlkwa}[1]{\textcolor[rgb]{0.161,0.373,0.58}{\textbf{#1}}}%
\newcommand{\hlkwb}[1]{\textcolor[rgb]{0.69,0.353,0.396}{#1}}%
\newcommand{\hlkwc}[1]{\textcolor[rgb]{0.333,0.667,0.333}{#1}}%
\newcommand{\hlkwd}[1]{\textcolor[rgb]{0.737,0.353,0.396}{\textbf{#1}}}%
\let\hlipl\hlkwb

\usepackage{framed}
\makeatletter
\newenvironment{kframe}{%
 \def\at@end@of@kframe{}%
 \ifinner\ifhmode%
  \def\at@end@of@kframe{\end{minipage}}%
  \begin{minipage}{\columnwidth}%
 \fi\fi%
 \def\FrameCommand##1{\hskip\@totalleftmargin \hskip-\fboxsep
 \colorbox{shadecolor}{##1}\hskip-\fboxsep
     % There is no \\@totalrightmargin, so:
     \hskip-\linewidth \hskip-\@totalleftmargin \hskip\columnwidth}%
 \MakeFramed {\advance\hsize-\width
   \@totalleftmargin\z@ \linewidth\hsize
   \@setminipage}}%
 {\par\unskip\endMakeFramed%
 \at@end@of@kframe}
\makeatother

\definecolor{shadecolor}{rgb}{.97, .97, .97}
\definecolor{messagecolor}{rgb}{0, 0, 0}
\definecolor{warningcolor}{rgb}{1, 0, 1}
\definecolor{errorcolor}{rgb}{1, 0, 0}
\newenvironment{knitrout}{}{} % an empty environment to be redefined in TeX

\usepackage{alltt} %[hyperref={pdfpagelabels=false}]
\usetheme{metropolis} 
\usepackage{bm}
\usepackage{cmbright}
\usepackage[utf8]{inputenc}
\usepackage[T1]{fontenc}
\usepackage{dcolumn}
\usepackage{booktabs}
\usepackage{amssymb}
\usepackage{amsmath}
\usepackage{tikz}
\usepackage{appendixnumberbeamer}
\usepackage{colortbl}
%\definecolor{lightgray}{rgb}{0.80,0.85,0.9}
%\definecolor{gray}{RGB}{155,155,155}

%\setbeameroption{hide notes}
%\setbeamertemplate{note page}[plain]

%\setbeamertemplate{footline}[frame number]
%\setbeamertemplate{navigation symbols}{}

\setbeamercolor{item}{fg=gray} % color of bullets
\setbeamercolor{subitem}{fg=gray}
\setbeamercolor{itemize/enumerate subbody}{fg=black}

%\setbeamercolor{frametitle}{fg=black}
%\setbeamercolor{frametitle}{bg=black!10!white}
\setbeamerfont{frametitle}{size=\Large}

\setbeamercolor*{title}{bg=white,fg=black}
\setbeamercolor{alerted text}{fg=orange}
\setbeamertemplate{blocks}[rounded][shadow=true]
%\addtobeamertemplate{block begin}{\pgfsetfillopacity{0.8}}{\pgfsetfillopacity{1}}
\setbeamercolor*{block title}{fg=white,bg=black!10!black}
\setbeamercolor*{block body}{fg=black!90!black,bg=black!10!white}

% reduce spacing between code chunk and printed console output
\setlength{\OuterFrameSep}{2pt}
\makeatletter
\preto{\@verbatim}{\topsep=-8pt \partopsep=-8pt }
\makeatother

\title{Generalised Linear Model: Bootstrapping and Permutations}
\author{Winnie Xia \\ \vspace{5mm} \scriptsize GV903: Advanced Research Methods, Week 17}
%\date{\includegraphics[width=0.35\textwidth]{logo-essex.pdf}}
\IfFileExists{upquote.sty}{\usepackage{upquote}}{}
\begin{document}

\def\solutions{} %% set to true
%\let\solutions\undefined %% set to false



\begin{frame}[plain]
\titlepage
\end{frame}

\begin{frame}
\scalebox{1.2}{
\begin{tabular}{ll}
1. & Bootstrapping
\end{tabular}
}
\end{frame}

\begin{frame}
\frametitle{Bootstrapping}
\begin{itemize}
\item \textbf{Definition}: It usually refers to a self-starting process that is to proceed without external input.
\item Applied to statistics: We sample with replace from the sample.
\end{itemize}
\end{frame}

\begin{frame}
\frametitle{Bootstrap}
Bootstrap is a desirable approach when:
\begin{itemize}
\item \textbf{the distribution of a statistic is unknown or complicated}.
\item \textbf{Reason:} bootstrap is a non-parametric approach and does not ask for specific distributions.
\item \textbf{the sample size is too small to draw a valid inference}.
\item \textbf{Reason:} it is a resampling method with replacement and recreates any number of resamples. 
\end{itemize}
\end{frame}

\begin{frame}
\frametitle{Let's break down "bootstrap"}
Bootstrap breaks down into the following steps:
\begin{itemize}
\item decide how many bootstrap samples to perform.
\item what is the sample size?
\item for each bootstrap sample:
\begin{itemize}
\item draw a sample with replacement with the chosen size
\item calculate the statistics of interests for that sample
\end{itemize}
\item calculate the mean of the calculated sample statistics. 
\end{itemize}
\end{frame}


\begin{frame}[fragile]
\frametitle{Bootstrapping Illustration in \texttt{R}}
We will try this with $n = 20$ for illustration. With larger samples, it will be asymptotically unbiased.

\end{frame}


\begin{frame}
\scalebox{1.2}{
\begin{tabular}{ll}
2. & Other Resampling Approaches
\end{tabular}
}
\end{frame}

\begin{frame}
\frametitle{Jackknife}
\textbf{It is a leave-one-out procedure.}

\end{frame}

\begin{frame}
\frametitle{Permutation}

\end{frame}

\begin{frame}
\frametitle{Permutation in \texttt{R}}
Let's create some data for this experiment.
\end{frame}


\begin{frame}
\frametitle{Permutation in \texttt{R}}
Let's create some data for this experiment.
\end{frame}

\end{document}
